\documentclass{article}
\usepackage{fancyhdr}
\usepackage{amsmath}
\usepackage{amssymb}
\usepackage{graphicx}
\usepackage{float}
\usepackage{subcaption}
\usepackage{hyperref}
\usepackage{listings}
\usepackage{color}
\usepackage{tikz}
\usepackage{pgfplots}
\usepackage{pgfplotstable}
\usepackage{listings}
\usepackage{color}

\definecolor{dkgreen}{rgb}{0,0.6,0}
\definecolor{gray}{rgb}{0.5,0.5,0.5}
\definecolor{mauve}{rgb}{0.58,0,0.82}

% Set up fancy header
\pagestyle{fancy}
\fancyhf{} % Clear default header and footer
\rhead{Keith Wesa} % Right header
\lhead{STAT 381 - Assingment 10} % Left header
\rfoot{Page \thepage} % Right footer

\author{Keith Wesa}
\title{STAT 381 - Written Homework 1}
\date{\today}

\begin{document}
\section{Problem 1}
\begin{itemize}
    \item[] Find $c$ so that the following is a valid probability mass function: $f(x) = cx \text{for} x = 5, 6, 9, 10$
    \item[] \textbf{Solution:}
    \item[] Since this is a probability mass function, we know that $\sum_{x=5}^{10} f(x) = 1$
\end{itemize}
    \begin{align*}
        \sum_{x=5}^{10} f(x) &= 1 \\
        \sum_{x=5}^{10} cx &= 1 \\
        c\sum_{x=5}^{10} x &= 1 \\
        c(5 + 6 + 9 + 10) &= 1 \\
        c(30) &= 1 \\
        c &= \frac{1}{30} \\
    \end{align*}



\end{document}