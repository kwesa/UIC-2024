\documentclass{article}
\usepackage{fancyhdr}
\usepackage{amsmath}
\usepackage{amsthm}
\usepackage{amssymb}
\usepackage{graphicx}
\usepackage{float}
\usepackage{subcaption}
\usepackage{hyperref}
\usepackage{listings}
\usepackage{color}
\usepackage{tikz}
\usepackage{pgfplots}
\usepackage{pgfplotstable}
\usepackage{listings}
\usepackage{color}
\usepackage{tikz}
\usepackage{amsthm}
\usepackage[margin=15mm]{geometry}

\definecolor{dkgreen}{rgb}{0,0.6,0}
\definecolor{gray}{rgb}{0.5,0.5,0.5}
\definecolor{mauve}{rgb}{0.58,0,0.82}

% Set up fancy header
\pagestyle{fancy}
\fancyhf{} % Clear default header and footer
\rhead{Keith Wesa} % Right header
\lhead{Stat 381 Review Exam 2:} % Left header
\rfoot{Page \thepage} % Right footer

\author{Keith Wesa}
\title{MAT 215 - Written Homework 1}
\date{\today}

\begin{document}
\section*{\textbf{Review:}}
\begin{enumerate}
    \item review the definitions of permutations / combinations and be able to identify them.
    be able to calculate probabilities with permutations / combinations.
    \begin{itemize}
        \item Permutations: $P(n,r) = \frac{n!}{(n-r)!}$
        \item If the problem says "order matters" then it is a permutation problem.
        \item Combinations: $C(n,r) = \frac{n!}{r!(n-r)!}$ 
        \item If the problem says "order doesn't matter" then it is a combination problem.
    \end{itemize}
    \item Know the addition rule.
    \begin{itemize}
        \item $P(A \cup B) = P(A) + P(B) - P(A \cap B)$
        \item If the problem is asking for the probability of A or B happening then it is an addition rule problem.
    \end{itemize}
    \item Know about intersection probabilities (AND) and conditional probabilities (GIVEN) and Homework
    they differ conceptually.
    \begin{itemize}
        \item Intersection probabilities are the probability of two events happening at the same time.
        \item $P(A \cap B) = P(A)P(B)$
        \item Conditional probabilities are the probability of an event happening given that another event has already happened.
        \item $P(A|B) = \frac{P(A \cap B)}{P(B)}$
    \end{itemize}
   \item How the difference between mutually exclusive and independent. Be able to calculate both of these options.
    \begin{itemize}
         \item Mutually exclusive events are events that cannot happen at the same time.
         \item $P(A \cap B) = 0$
         \item Independent events are events that do not affect each other.
         \item $P(A|B) = P(A)$
    \end{itemize}
    \item Be able to calculate probabilities when you don't have things that are mutually exclusive or independent 
    when having trouble draw a picture. 
    \item Know how to use the compliment rule when trying to calculate greater than or equal to probabilities.
    \begin{itemize}
        \item $P(A^c) = 1 - P(A)$
    \end{itemize}
    \item Be able to calculate probabilities using Bayes' Rule Tree Diagrams nd how to draw them. 
    \begin{itemize}
        \item $P(A|B) = \frac{P(B|A)P(A)}{P(B)}$
        \item Bayes' Rule is used when you are given the probability of B given A and you need to find the probability of A given B.
        \item Tree diagrams are used to help visualize the problem.
        \item The first branch is the probability of A happening.
        \item The second branch is the probability of B happening given that A has already happened.
        \item The third branch is the probability of B happening given that A has not happened.
    \end{itemize}
\end{enumerate}


\end{document}
