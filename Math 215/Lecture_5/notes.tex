%\input{C:\NoteSchema\Templates\HomeworkTemp_151.tex}
\documentclass[12pt]{article}
\usepackage{amsmath}
\usepackage{amssymb}
\usepackage{fancyhdr}
\usepackage{lastpage}
\usepackage{mathrsfs}

\author{Keith Wesa}
\title{Math 215 Lecture 5 Notes}
\date{\tikzaliascoordinatesystem{new name}{old name}
\begin{document}

% % \maketitleff
\section{Definitions}
\begin{itemize}
    \item[] \textbf{Collection}: A set of objects, usually denoted with capital letters.
    \begin{itemize}
        \item[i] \textbf{Roster}: $A = \{1, 2, 3, 4, 5\}$
        \item[ii] \textbf{Sentence}: eg. a set of positive integers
        \item[iii] \textbf{Set-Builder}: $B = \{x \in \mathbb{Z} | x \leq 5\}$     
    \end{itemize}
    \item[] \textbf{Element}: An object in a collection, usually denoted with lowercase letters.

\end{itemize}
\section{Notes}
\begin{itemize}
    \item[Q:] What does result mean?
    \begin{itemize}
        \item[] what would we need to do to prove it?
        \begin{itemize}
            \item[(i)] For all integers
            \[ 
                n \geq 5, 2^n > n^2
            \]
            \item[(ii)] There is an integer  $\alpha$ such that:
            \[
                \alpha^2 + 29\alpha + 209 \geq 0
            \]
        \end{itemize}
    \end{itemize}
\end{itemize}


\end{document}
