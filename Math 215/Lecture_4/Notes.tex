\documentclass{article}

% Packages
\usepackage{amsmath} % For mathematical symbols and equations
\usepackage{amssymb} % For additional mathematical symbols
\usepackage{enumerate} % For customizing enumeration
\usepackage{geometry} % For adjusting page margins
\usepackage{graphicx} % For including images
\usepackage{hyperref} % For adding hyperlinks
\usepackage{listings} % For including code snippets
\usepackage{tikz} % For creating diagrams and graphs

% Page setup
\geometry{margin=1in} % Adjust margins as needed

% Title
\title{Discrete Mathematics Notes}
\author{Keith Wesa}
\date{\today}

\begin{document}

\maketitle

\section{Introduction}
\begin{itemize}
    \item Today we will be learning about predicate logic.
\end{itemize}
\section{Logic}
\begin{itemize}
    \item Logic is the study of reasoning.
    \item It is used in mathematics, philosophy, and computer science.
    \item Logic is used to determine whether an argument is valid or not.
    \item An argument is a sequence of statements that end with a conclusion.
    \item A statement is a sentence that is either true or false.
    \begin{itemize}
        \item Examples of statements:
        \begin{itemize}
            \item The sky is blue.
            \item The sky is red.
            \item The sky is green.
        \end{itemize}
        \item Examples of non-statements:
        \begin{itemize}
            \item What time is it?
            \item Go to the store.
            \item $x + 2 = 5$
        \end{itemize}
    \item A valid argument is one where the conclusion is true if the premises are true.
    \item An invalid argument is one where the conclusion is false if the premises are true.
    \item A tautology is a statement that is always true.
    \item A contradiction is a statement that is always false.
    \item A contingency is a statement that is neither a tautology nor a contradiction.
    \item A proposition is a statement that is either true or false.
    \item logical equivalence
    \begin{itemize}
        \item Two statements are logically equivalent if they have the same truth value.
        \item Example: $p \land q$ is logically equivalent to $q \land p$.
        \item Example: $p \lor q$ is logically equivalent to $q \lor p$.
        \item Example: $p \rightarrow q$ is logically equivalent to $\neg p \lor q$.
        \item Example: $p \leftrightarrow q$ is logically equivalent to $(p \rightarrow q) \land (q \rightarrow p)$.
        \item Example: $\neg(p \land q)$ is logically equivalent to $\neg p \lor \neg q$.
        \item Example: $\neg(p \lor q)$ is logically equivalent to $\neg p \land \neg q$.
        \item Example: $\neg(p \rightarrow q)$ is logically equivalent to $p \land \neg q$.
    \end{itemize}
    \item De Morgan's Laws
    \begin{itemize}
        \item $\neg(p \land q)$ is logically equivalent to $\neg p \lor \neg q$.
        \item $\neg(p \lor q)$ is logically equivalent to $\neg p \land \neg q$.
    \end{itemize}
\end{itemize}

\section{Sets and Relations}

% Your notes go here

\section{Functions}

% Your notes go here

\section{Counting and Probability}

% Your notes go here

\section{Graph Theory}

% Your notes go here

\section{Conclusion}

% Your notes go here

\end{document}
