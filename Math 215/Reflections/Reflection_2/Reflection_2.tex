\documentclass{article}
\usepackage{fancyhdr}
\usepackage{amsmath}
\usepackage{amsthm}
\usepackage{amssymb}
\usepackage{graphicx}
\usepackage{float}
\usepackage{subcaption}
\usepackage{hyperref}
\usepackage{listings}
\usepackage{color}
\usepackage{tikz}
\usepackage{pgfplots}
\usepackage{pgfplotstable}
\usepackage{listings}
\usepackage{color}
\usepackage{tikz}
\usepackage{amsthm}
\usepackage{soul}
\usepackage[margin=15mm]{geometry}

\definecolor{dkgreen}{rgb}{0,0.6,0}
\definecolor{gray}{rgb}{0.5,0.5,0.5}
\definecolor{mauve}{rgb}{0.58,0,0.82}
\hypersetup{colorlinks=true,linkcolor=blue, linktocpage}
% Set up fancy header
\pagestyle{fancy}
\fancyhf{} % Clear default header and footer
\rhead{Keith Wesa} % Right header
\lhead{MATH 215 Reflections} % Left header
\rfoot{Page \thepage} % Right footer

\author{Keith Wesa}
\title{MAT 215 - Written Homework 1}
\date{\today}

\begin{document}
\section*{\text{Prompt 1}}
Academic misconduct is a very relevant concern in all courses. For us, this is
concern revolves around the weekly assignments and midterm. It can sometimes,
however, be unclear what exactly constitutes a misconduct. Understanding what
constitutes academic misconduct is crucial for maintaining fairness and honesty
within academic environments. It ensures that students are evaluated based on
their own efforts, promoting a level playing field for all. I would like to gauge your
thoughts on misconduct in this class.
\newline
\newline
Please respond to the following questions below:
\subsection*{Questions}
\begin{itemize}
    \item[i.] Bob and Alice work together on an assignments
    \item[] \textbf{Response:} This is not academic misconduct. It's fine for a couple of people to work
    together on an assignment. In fact, it should be encouraged as it promotes collaboration and teamwork.
    \item[ii.] Alice can see Bob's paper during the midterm, and copies the answers she doesn't know. 
    \item[] \textbf{Response:} This is academic misconduct. Alice is stealing someone elses work, and isn't 
    being evaluated based on her own efforts. It's also unfair to Bob, who is getting taken advantage of.
    \item[iii.] Alice gets advice from the professor regarding the assignment and shares this
    advice with Bob.
    \item[] \textbf{Response:} This is not academic misconduct. Theres nothing wrong with getting advice and sharing it with 
    another student, as long as the work is still your own. I think if I were in this situation, I would encourage Bob to 
    also seek advice from the professor, and take ownership of his studies. 
    \item[iv.]  Bob and Alice agree that they will take turns completing the assignment,
    and let the other copy their solutions.
    \item[] \textbf{Response:} This is academic misconduct. This is effectively the same as copying someone elses work.
    \item[v.] Bob asks his friend Carl, who has already taken this course, to do the assignment for him.
    \item[] \textbf{Response:} This is academic misconduct. This is stealing someone elses work. It's also not good to do this because
    Bob isn't learning anything, and will be at a disadvantage in the future.
    \item[vi.]  Alice can see Bob's paper during the midterm, and let's him know that he
    has an error.
    \item[] \textbf{Response:} This is academic misconduct. Alice is giving Bob an unfair advantage. It also makes it harder for Bob to know 
    what he needs to work on in the future. There's nothing wrong with getting the wrong answer, as long as you learn from it.
    \item[vii.]  Before submitting the assignment, Alice notices that Bob has an error and
    explains how to fix it.
    \item[] \textbf{Response:} This is not academic misconduct. As long as Bob is the one who makes the changes, there's nothing wrong with someone 
    pointing out an error.
\end{itemize}
\section*{\text{Prompt 2}} 

We have or will have seen already in Assignment 3 that generative AI, like Chat-GPT, is not entirely reliable for producing proofs in mathematics. That does not
mean however, that the use of such a tool has no place in the class or more broadly
our lives. I use it for trivial tasks, like when deciding what to cook or to ask random questions, for personal tasks, such as generating a detailed description for a
location in a game of Dungeons and Dragons, and for works tasks, such as having it
parse and summarize the introduction of an uploaded research paper to determine
if it is worth reading through proper.
\newline
\newline
Please respond to the following questions below:
\subsection*{Questions}
\begin{itemize}
    \item[] How do you make use of generative AI tools in your life? If you do not use them
    right now, have you seen a novel use of it by someone else? In whats ways do you
    think they could be beneficial?
    \item[] \textbf{Response:} I use generative AI tools for several things. My main use is to help me code, by using tools like
    github copilot. I also like bings built in AI, which is nice for getting quick explanations for very specific and low level questions.
    I think generative AI tools are beneficial because, when using them it's like having a brainstorming buddy there helping you. However,
    they are just tools, and make a lot of mistakes. I also think that there are many ways in which they are not beneficial, such as how they are 
    gathering data. For example, it's very important for me to upload my code to github, so that the AI can learn from it. This is a privacy concern,
    as well as a concern for intellectual property. I also really don't like how "Art" is being made by AI. I think it's important for art to be a human.
    What I really like it for, is dumb stuff like taking old text from old video games and giving it text to speech that isn't garbage.
\end{itemize}
\section*{\text{Prompt 3}}
This prompt is not worth any marks and your answer will not count towards the
grade for this reflection. It is an opportunity for you to say hello, to tell me something interesting that you learned so far in the course, to give me some feedback,
etc.
\newline
\newline
Is there anything else you would like to share about your experiences with the course so far, good, bad, or neutral?
\subsection*{Response}
\begin{itemize}
    \item[] I'm really enjoying the course so far. I'm learning something similar in a course called CS 151 however 
    we're going over the broad strokes in that course, and we're going over the details in this course. What I like most about the
    course is that it's very laid back, and there isn't a ton of pressure put on the students. It's a nice reprieve from the other courses
    I'm taking, which are very demanding. I also really like the way the course is structured. I think that it values the students time, and
    I really think that other courses can learn a lot from this structure. 
\end{itemize}
\end{document}