\documentclass{article}
\usepackage{graphicx}
\usepackage{tikz}
\usepackage{pgfplots}
\pgfplotsset{compat=1.18}
\usepackage{graphicx}
\usepackage{tikz}
\usepackage{pgfplots}

\title{Math 215 Reflection 1}
\author{Keith Wesa}
\date{\today}

\begin{document}
\maketitle 

\section{Prompt 1}

We are fortunate to have a small class setting for this course. It will be nice and beneficial
to get to know a bit about everyone.
\newline

Please respond to following questions below:
\newline

What year of study are you in? What program are you taking or considering? Tell me
something constructive that you did over the past year and/or something constructive you
intend to do this year (i.e. new hobbies or skills, focusing on mental health, etc.).
\subsection{Response}

    I'm a jounior in a computer science program. The constructive thing that I've done over the
past year that isn't related to school or work is that I've been volunteering at the Illinois
Railway Museum. I installed a train power station, and also helped rewire a old metra car.As 
well as gotten the trains ready for special events that are held for the public. 
\section{Prompt 2}

One of my favourite poems is “When I Heard the Learn’d Astronomer” by Walt Whitman.
I am quite notably not an expert in poetry, but I think the actual meaning of the poem is
that nature can only be understood by experiencing it rather than by sitting in a classroom.
I like to interpret this poem as saying that it is easy to lose passion in a subject when it
is presented in a mundane fashion or lacks motivation
\newline
please read and respond to the poem below:
\begin{center}
    When I Heard the Learn’d Astronomer
    \newline
    By: Walt Whitman
    \newline 
    \newline
    When I heard the learn’d astronomer,
    \newline
    When the proofs, the figures, were ranged in columns before me,
    \newline
    When I was shown the charts and diagrams, to add, divide, and measure them,
    \newline
    When I sitting heard the astronomer where he lectured with much applause in the
    lecture-room,
    \newline
    How soon unaccountable I became tired and sick,
    \newline
    Till rising and gliding out I wander’d off by myself,
    \newline
    In the mystical moist night-air, and from time to time,
    \newline
    Look’d up in perfect silence at the stars.

\end{center}

What are your thoughts on the poem? What expectations do you have for the course?
What would help to make this course interesting for you?
\subsection{Response}
My thoughts on this poem is that it highlights the need for recentering and remebering why
we study, why we learn, that all the work seems to be for nothing if not grounded in the fundimental
love for what we are learning. The author derives meaning from the mundane, "perfect silence of the stars" 
and that meaning makes the rest of his work that was making him "tired and sick" worth it.
\newline

I expect this course to help encourage my love for math, and to help introduce me to concepts that are outside 
of the current way that I think about math. 
\newline 

What would make this course more insteresting to me is, reflecing on where these concepts came from, and how they 
are applied in the real world. This will help me understand how to better apply these concepts in my own life, and 
become a better programmer. 

\section{Prompt 3}

This course is notably different than many of the other (math) courses you might take in
during your time as an undergraduate student. Rather than learning (mathematical) content, we are focusing on the ideas and methods of proofs. Such a topics do not (necessarily)
have clear-cut real-world examples or applications.
\newline

Please respond to following questions below:
\newline

Why do we need to prove things?

\subsection{Response}
I think I'll answer this question with a story. The Therac-25 was a radiation therapy machine that was used in the 1980s.
It was a machine that was used to treat cancer patients. This machine took the lives of 3 people, and injured 3 others. The reason for this was that the machine was not properly tested.
The machine had a code that would resolve to an error. When the error occured the machine would give a lethal dose of radiation to the patient.
How this applies to mathematical proofs is that we need to prove things are working correctly, it's important to invalidate our assumptions. Sometimes
our assumptions are wrong, and we need to prove that they are wrong. The implications of this could mean the matter of life or death in some cases. 


\end{document}