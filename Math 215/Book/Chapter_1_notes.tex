\documentclass{article}

% Packages
\usepackage{amsmath} % For mathematical symbols and equations
\usepackage{amsthm} % For theorem environments
\usepackage{enumitem} % For customizing lists

% Theorem Environments
\newtheorem{theorem}{Theorem}[section]
\newtheorem{lemma}[theorem]{Lemma}
\newtheorem{corollary}[theorem]{Corollary}
\newtheorem{definition}[theorem]{Definition}
\newtheorem{example}[theorem]{Example}
\newtheorem{remark}[theorem]{Remark}

% Custom Commands
% Remove the conflicting \newcommand{\proof} line
% \newcommand{\qed}{\hfill$\square$}

% Document
\begin{document}

\title{Notes on Chapter 1 of \textit{Proof and the Art of Mathematics}}
\author{Keith Wesa}
\date{\today}
\maketitle

\section{Chapter 1: Introduction}
    \subsection{Definitions}
    \begin{itemize}
        \item \textbf{Mathematical Proof} is a convincing argument 
        (within the context of mathematics) that a mathematical statement is true.
        \item \textbf{Theorem} is a mathematical statement that has been proved.
        \item \textbf{Lemma} is a theorem that is used as a stepping stone to prove other theorems.
        \item \textbf{Corollary} is a theorem that follows directly from another theorem.
        \item \textbf{Definition} is a statement that gives the precise meaning of a mathematical term.
        \item \textbf{Example} is a mathematical statement that illustrates a theorem or definition.
        \item \textbf{Incommensurable} is a term used to describe two lengths that do not have a common unit of measure.
        \item \textbf{Contradiction} is a statement that is false.
    \end{itemize}
\subsection{The Pythagorean's}
    \begin{itemize}
        \item Pythagorean's believed that all numbers could be expressed as a ratio of integers.
        \begin{itemize}
            \item This is not true, as $\sqrt{2}$ is irrational.
            \item This is not true, as $\pi$ is irrational.
            \item The Pythagoreans discovered in the fifth century B.C that the side and diagonal
            of a square are not commensurable. By that I mean that the square has o common unit of measure.
            \item If you divide the side of the square into ten equal units, then the diagonal will be little more than 
            fourteen of those units. If you were to divide it into 1,000 units, then it would be little more than 1,414 units.
            \item $\sqrt{2} = 1.41421...$ which is irrational.     
    \end{itemize}
\begin{theorem}
    Statement of the theorem goes here.
\end{theorem}

\begin{proof}
    Proof of the theorem goes here.
\end{proof}

% Add more sections, theorems, lemmas, examples, etc. as needed

\end{document}
