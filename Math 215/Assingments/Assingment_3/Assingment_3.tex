\documentclass{article}
\usepackage{fancyhdr}
\usepackage{amsmath}
\usepackage{amsthm}
\usepackage{amssymb}
\usepackage{graphicx}
\usepackage{float}
\usepackage{subcaption}
\usepackage{hyperref}
\usepackage{listings}
\usepackage{color}
\usepackage{tikz}
\usepackage{pgfplots}
\usepackage{pgfplotstable}
\usepackage{listings}
\usepackage{color}
\usepackage{tikz}
\usepackage{amsthm}
\usepackage[margin=15mm]{geometry}

\definecolor{dkgreen}{rgb}{0,0.6,0}
\definecolor{gray}{rgb}{0.5,0.5,0.5}
\definecolor{mauve}{rgb}{0.58,0,0.82}

% Set up fancy header
\pagestyle{fancy}
\fancyhf{} % Clear default header and footer
\rhead{Keith Wesa} % Right header
\lhead{MAT Written Assingment 2:} % Left header
\rfoot{Page \thepage} % Right footer

\author{Keith Wesa}
\title{MAT 215 - Written Homework 1}
\date{\today}

\begin{document}
\section*{Question 1}
\textbf{For each set $S$ given below, describe the set using other two notations: 
sentence notation, roster notation, or set-builder notation.} (Hint: to find a suitable sentence
for b. and c, think about what the set is describing. The set-builder form for a and b is not unique.)
(Note: When describing the set in roster notation, you need to provide enough elements to make 
the set in roster notation, you need to provide enough elements to make the set clear.) \textbf{[6 Marks]}
\begin{itemize}
    \item[a.] The set $S$ of integers divisible by 3 and 5.
    \begin{itemize}
        \item[] \textbf{Roster Notation: } $S = \{15, 30, 45, 60, 75, 90\}$
        \item[] \textbf{Set-Builder Notation: } $S = \{n \in \mathbb{Z} : n = 3k \land n = 5p\}$
        \begin{proof}
        \begin{center} 
        \begin{tabular}{|c|c|c|c|} 
        \hline $k$ & $p$ & $3k$ & $5p$ \\ 
        \hline 
        1 & 1 & 1 & 1 \\
        1 & 0 & 0 & 0  \\
        0 & 1 & 0 & 0 \\
        0 & 0 & 0 & 0  \\
\hline 
\end{tabular} 
\end{center}
        \item[] In order to satisfy the condition for $S$, $n$ must be a multiple of both 3 and 5. The statement would be false any other way.
        \end{proof}

    \item[b.] $S =\{1,2,3,4,6,12\}$
    \item[] \textbf{Sentence Notation: } The set of all positive integers that are factors of 12.
    \item[] \textbf{Set-Builder Notation: } 
    \begin{align*}
        S &=\{1,2,3,4,6,12\} \\
        P &= \{12, 6, 4, 3, 2, 1\}\\ 
        S &= \{\forall n \in S(n) \forall p \in P(p) : n = \frac{12}{p} \} \\
    \end{align*}
    \begin{itemize}
        \item I thought this was an interesting question, and I couldn't figure out how to properly write out an answer for it
        what I find interesting is as the number $n$ increases then number for $p$ decreases. I'm not sure how to properly write this out in set-builder notation.
        So, I just started throwing math at it. 
    \end{itemize}
    \item[]
    \item[c.] $S = \{4^k : k \in \mathbb{Z}\}$
    \item[] \textbf{Sentence Notation: } The set of integers that raise 4 to the power of $k$.
    \item[] \textbf{Set-Builder Notation: } $S = \{1, 4, 16, 64, 256\}$
\end{itemize} 
\end{itemize} 

\end{document}
