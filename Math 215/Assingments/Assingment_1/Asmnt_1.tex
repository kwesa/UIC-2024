\documentclass[12pt]{article}
\usepackage{mathbbol}
\usepackage{amsmath}
\usepackage{graphicx}
\usepackage{mathbbol}
\usepackage{amsmath}
\usepackage{graphicx}


\begin{document}

\title{Assignment 1}
\author{Keith Wesa}
\date{\today}

\maketitle

\section{Question 1}

The most common first experience students have with “proofs” is by proving
trigonometric identities. It is important to understand that when we prove trigonometric identities, we are not just pushing around symbols until both sides are equal.
We need to be wary of when our equality makes sense.


\begin{itemize}
    \item[a.] Use trigonometric identities to prove that:

    \begin{align}
        
        $\sec(x) - \tan(x)\sin(x) = \cos(x)$

        $\frac{1}{\cos(x)} - \frac{\sin(x)}{\cos(x)}\sin(x) = \cos(x)$

        $\frac{1}{\cos(x)} - \frac{\sin^2(x)}{\cos(x)} = \cos(x)$

        $\frac{1 - \sin^2(x)}{\cos(x)} = \cos(x)$

        $\frac{\cos^2(x)}{\cos(x)} = \cos(x)$

        $\cos(x) = \cos(x)$ 

    \end{align}

\end{itemize}
\begin{itemize}
    \item[b.] For what values of x is the left-hand side defined? For what values of x is
    the right-hand side defined? (i.e. What are the domains?)

    
    
    The left-hand side is defined for all values of x except for $\frac{\pi}{2} + n\pi$ where n is an integer.
    We know this because $\sec(x)$ is undefined for $\frac{\pi}{2} + n\pi$ where n is an integer. This is because 
    the quotient identity for $\sec(x) = \frac{1}{\cos(x)}$ and $\cos(\frac{\pi}{2} + n\pi) = 0$. However, due to the
    properties of functions, and the fact it is not simplified, when you simplify it resolves to $\cos(x)$ which is defined for all values of x.


    The right-hand side is defined for all values of x.

    \item[c.] Write a statement which clearly indicated when the trigonometric identity is
    true.

\begin{center}
    $\sec(x) - \tan(x)\sin(x) = \cos(x)$   
\end{center}
is true for all real numbers. $ x \in \mathbb{R}$
\end{itemize}


\section{Question 2}

In lecture, we proved that $\sqrt{2}$ is an irrational number. In our proof, we appealed
to several facts that we probably believe are true (whether we were told that they
are true, or because we have seen a sufficient plethora of examples to convince
ourselves). Whenever possible, we want to know why the facts that we are using
are true (although even for a veteran mathematician, it is impossible to know why
all facts are true).

\begin{itemize}
 \item[2.] If $p^2$ is even, then $p$ is even.
\end{itemize}
\begin{itemize}
    \item [a.] What does it mean for a number to be even? What does it mean for a number
    to be odd?

        \begin{proof}
            A number is even if it is divisible by 2.
            A number is odd if it is not divisible by 2.
        \end{proof}
            
    \item [b.]  Consider the numbers 16, 36, 64, 100. What are their square roots? Are
    they even?

        \begin{proof}
            The square roots of 16, 36, 64, 100 are 4, 6, 8, 10 respectively. They are all even.
        \end{proof}

    \item [c.] What is a general expression to represent even numbers? What is a general
    expression to represent odd numbers?

        % Rest of the document...

        \begin{proof}
            A general expression to represent even numbers is $2n$ where $n \in \mathbb{Z}$.
            A general expression to represent odd numbers is $2n + 1$ where $n \in \mathbb{Z}$.
        \end{proof}

    \item [d.] Suppose that $p$ is an odd number. Find an expression for $p^2$. Is $p^2$ even or odd?
            
            \begin{align}
                $p^2 = (2p + 1)^2$ where $n \in \mathbb{Z}$.

                $p^2 = 4p^2 + 4p + 1$ where $n \in \mathbb{Z}$.

                $p^2 = 4(p^2 + p) + 1$ where $n \in \mathbb{Z}$.
                
                If you put any value of $n$ into the equation, it will always be odd.
            \end{align}
    \item [e.] What can you conclude about the theorem based on your answer to part (d)?
                
        I can conclude that the theorem that if $p^2$ is even, then $p$ is even is true.

    \item [f.] using similar logic and Theorem 1, prove that the following related fact:
    
    \subitem if $p^4$ is even, then $p$ is even. (Hint: $p^4 = (p^2)^2$)
    \begin{align}

        even:


        $p^4 = (p^2)^2$

        $p^4 = (2p^2)^2$ where $n \in \mathbb{Z}$.

        $p^4 = 4p^4 $ where $n \in \mathbb{Z}$.

        so $p^4$ is even.

        odd:

        $p^4 = [4(p^2 + p) + 1]^2$ where $n \in \mathbb{Z}$.

        $p^4 = 16(p^2 + p)^2 + 8(p^2 + p) + 1$ where $n \in \mathbb{Z}$.

        What is in the parenthesis is always even, so if we add 1 to it, it will always be odd.

    
    \end{align}

\end{itemize}


% attach image here!

\section{Question 3}

Once a fact has been proven, a natural next step is to look for generalizations of
that fact. It is sometimes possible to reuse the proof method (with some small
changes) to prove one of the generalizations. In some extreme cases, a similar proof
method can be used to prove seemly unrelated facts.

\begin{itemize}
    \item [a.] Prove that $\sqrt[4]{2}$ is irrational.

    \begin{proof}
        \begin{align}
        Given: 

        The Irrationality of the $\sqrt{2}$ is found by:

        $\sqrt{2} = \frac{p}{q}$ with no common factors and $q \neq 0$  $k=common factor$
        
        $p^2 = 2q^2$

        $p^2$ is an even number 

        If $p^2$ is even then $p$ is also even 


        $2q^2 = p^2 = (2k)^2 = 4k^2$

        $q^2 = 2k^2$

        In context of the $\sqrt[4]{2}$ we can take it above.

        $2q^4 = p^4 = (2k)^4 = 16k^4$

        Given that $q^2$ is even, and $q$ is even. Both would contradict each other because there would be an even common factor. 
        This is a proof by contradiction.
        
        \end{align}
    
    \end{proof}

\end{itemize}


\end{document}