\documentclass{article}

% Packages
\usepackage{amsmath} % For mathematical symbols and equations
\usepackage{amsthm} % For theorem environments
\usepackage{enumitem} % For customizing lists

% Theorem environments
\newtheorem{theorem}{Theorem}
\newtheorem{lemma}[theorem]{Lemma}
\newtheorem{corollary}[theorem]{Corollary}
\newtheorem{definition}{Definition}

% Custom commands
\newcommand{\proofcommand}{\noindent\textbf{Proof.}\hspace{0.5em}}
% \newcommand{\qed}{\hfill$\square$}

% Document information
\title{Notes for Proofs Class: Proving that $\sqrt2$  is irrational}
\author{Your Name}
\date{\today}

\begin{document}


\maketitle

\section{Learining Objectives}
    \begin{itemize}
        \item describe the proof of several elementary facts.
        \item apply similar proof methods to similar elementry facts.
    \end{itemize}

    Before we delve into the finer details of logic and proofs, it will be beneficial to see some 
    examples of pfoofs to help motivate the ideas we are learning. While one might think that Proofs
    all have to do with high-level abstract math, there are actually many elementrory facts that have 
    interesting and inspired proofs. 

\section{Introduction}

     The Pythagoreans discovered that the diagonal of a square with side length 1 has length $\sqrt2$.
     They also discovered that $\sqrt2$ is not a rational number.
     This was a shock to them, because they believed that all numbers were rational.
     This discovery led to the discovery of irrational numbers.
     In plaine words, an irrational number is a number that cannot be expressed as a fraction of two integers.
    

% Your notes go here

\section{Theorems}

\begin{theorem}
    Suppose toward a contradiction that $\sqrt2$ is rational. This means that $\sqrt2 = \frac{p}{q}$
    for some integers $p$ and $q$ with no common factors. Then $2 = \frac{p^2}{q^2}$, so $2q^2 = p^2$.
    This means that $p^2$ is even, so $p$ is even. So $p = 2k$ for some integer $k$. Then $2q^2 = (2k)^2 = 4k^2$,
    so $q^2 = 2k^2$. This means that $q^2$ is even, so $q$ is even. But this means that $p$ and $q$ have a common
    factor of 2, which contradicts our assumption. Therefore, $\sqrt2$ is irrational.
     
    A number $\alpha$ is rational if $\alpha = \frac{p}{q}$ for some integers $p$ and $q$ i.e (0,$\pm1$,$\pm2$,$\pm3$,...),
    with $q \neq 0$.

    A number $\alpha$ is irrational if $\alpha$ is not rational, i.e., $\alpha \neq \frac{p}{q}$ for any integers $p$ and $q$.

    In other words. The theorem states that $\sqrt{2}$ cannot be expressed as a fraction of two integers.

\end{theorem}

  \begin{proof}
    Suppose toward a contradiction that $\sqrt2$ is rational. This means that $\sqrt2 = \frac{p}{q}$
    for some integers $p$ and $q$ with no common factors. Then $2 = \frac{p^2}{q^2}$, so $2q^2 = p^2$.
    This means that $p^2$ is even, so $p$ is even. So $p = 2k$ for some integer $k$. Then $2q^2 = (2k)^2 = 4k^2$,
    so $q^2 = 2k^2$. This means that $q^2$ is even, so $q$ is even. But this means that $p$ and $q$ have a common
    factor of 2, which contradicts our assumption. Therefore, $\sqrt2$ is irrational.
\end{proof}

    This proof is an example of "proof by contradiction", where we assume the conclusion i.e. ($\sqrt{2}$ is irrational) is false,
    and aim to find an impossilbe consequence  i.e ($\frac{p}{q}$ is both in loweest terms and not in the lowest terms).



\section{Definitions}

\begin{definition}
    % Definition
\end{definition}

% More definitions

\end{document}
