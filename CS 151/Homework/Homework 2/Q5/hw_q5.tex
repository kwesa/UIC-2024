\documentclass{article}
\usepackage{fancyhdr}
\usepackage{amsmath}
\usepackage{amsthm}
\usepackage{accents}
\usepackage{amssymb}
\usepackage{graphicx}
\usepackage{float}
\usepackage{subcaption}
\usepackage{hyperref}
\usepackage{listings}
\usepackage{color}
\usepackage{tikz}
\usepackage{pgfplots}
\usepackage{pgfplotstable}
\usepackage{listings}
\usepackage{color}
\usepackage{tikz}
\usepackage{amsthm}
\usepackage{soul}
\usepackage[margin=15mm]{geometry}
\usepackage[makeroom]{cancel}

\definecolor{dkgreen}{rgb}{0,0.6,0}
\definecolor{gray}{rgb}{0.5,0.5,0.5}
\definecolor{mauve}{rgb}{0.58,0,0.82}
\hypersetup{colorlinks=true,linkcolor=blue, linktocpage}
% Set up fancy header
\pagestyle{fancy}
\fancyhf{} % Clear default header and footer
\rhead{Keith Wesa} % Right header
\lhead{CS 151 Homework 2:} % Left header
\rfoot{Page \thepage} % Right footer

\author{Keith Wesa}
\title{MAT 215 - Written Homework 1}
\date{\today}

\begin{document}
\section*{Question 2.2}
\subsection*{The problem}
\begin{itemize}
    \item[Q2.2] For any integers $x$, $y$, and $z$ with $x \neq y$, if $z$ is divisible by both $x - y$, then $z$ is divisible by $y - x$.
    \item[] \textbf{Define: } Divisibility: $a$ is divisible by $b$ if there exists an integer $c$ such that $a = bc$.
    \item[] \textbf{Compound Proposition:} $\forall x, y, z \in \mathbb{Z}, x \neq y (z = (x - y)k \rightarrow z = (y - x)l)$
    \item[] \textbf{Simple Proposition:} $z = (x - y)k  = p$ and $z = (y - x)l = q$
    \item[] \textbf{Logical Statement:} $p \rightarrow q$
    \item[] \textbf{Thoughts on the problem: } We can prove this problem by contradiction. We can assume that $z$ is divisible by $x - y$ and $z$ is not divisible by $y - x$. 
    Then we can show that this assumption leads to a contradiction.
    \item[] \textbf{Contradiction:} $p \land \lnot q$ to prove we'll have to show that $p \land \lnot q \equiv \text{False}$
\end{itemize}
\subsection*{Proof by Contradiction}
\begin{proof}
    
    \begin{itemize}
        \item[]Using the definition of divisibility assume that $z$ is divisible by $x - y$ and $z$ is not divisible by $y - x$. Then there exists an integer $k$ such that $z = (x - y)k$ and there exists an integer $l$ such that $z \neq (y - x)l$.
        \item[] \textbf{Defining Set: }$x, y, z, k, l \in \mathbb{Z}$ and $x \neq y$
        \begin{align*}
        z &= (x - y)k \\
        z &= (y - x)l \\
        (x - y)k &= (y - x)l \\
        k &= \frac{(y - x)l}{(x - y)} \\
        \text{ Substituting } k \text{ into the first equation: } \\
        z &= (x - y)\frac{(y - x)l}{(x - y)} \\
        z &= \cancel{(x - y)}\frac{(y - x)l}{\cancel{(x - y)}} \\ 
        z &= (y - x)l
    \end{align*}
    \item[] Since $z = (y - x)l$, and $z = (x - y)k$ are divisible, then $z$ is divisible by both $x - y$ and $y - x$.
    \item[] This is a contradiction. Therefore, $z$ is divisible by $y - x$.
    \end{itemize}
\end{proof}
\end{document}