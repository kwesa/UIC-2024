\documentclass{article}
\usepackage{fancyhdr}
\usepackage{amsmath}
\usepackage{amsthm}
\usepackage{accents}
\usepackage{amssymb}
\usepackage{graphicx}
\usepackage{float}
\usepackage{subcaption}
\usepackage{hyperref}
\usepackage{listings}
\usepackage{color}
\usepackage{tikz}
\usepackage{pgfplots}
\usepackage{pgfplotstable}
\usepackage{listings}
\usepackage{color}
\usepackage{tikz}
\usepackage{amsthm}
\usepackage{soul}
\usepackage[margin=15mm]{geometry}

\definecolor{dkgreen}{rgb}{0,0.6,0}
\definecolor{gray}{rgb}{0.5,0.5,0.5}
\definecolor{mauve}{rgb}{0.58,0,0.82}
\hypersetup{colorlinks=true,linkcolor=blue, linktocpage}
% Set up fancy header
\pagestyle{fancy}
\fancyhf{} % Clear default header and footer
\rhead{Keith Wesa} % Right header
\lhead{CS 151 Homework 2:} % Left header
\rfoot{Page \thepage} % Right footer

\author{Keith Wesa}
\title{MAT 215 - Written Homework 1}
\date{\today}

\begin{document}
\section*{Question 1}
Write each of the following arguments in argument form (i.e., using propositions, predicates, quantifiers, and logical operators). 
Then, use the rules of inference to prove that each argument is valid. You must indicate the name of the rule used in each step of the proof.
\subsection*{The problem}
\begin{itemize}
    \item[Q1.1] If I am in New York, then I won't get a hot dog. I am in New York or Chicago.
     I will get some mustard and a hot dog. Therefore, I am in Chicago or Germany
     \item[] let c be "I am in Chicago", n be "I am in New York", g be "I am in Germany", h be "I will get a hot dog", 
        m be "I will get some mustard"
    \item[] \textbf{Compound Logic Form: } $(n \rightarrow \neg h) \land (n \lor c) \land (m \land h) \rightarrow (c \lor g)$
    \item[] \textbf{Argument Form: }
    \begin{equation*}
        \begin{aligned}
            & n \rightarrow \neg h \\
            & n \lor c \\
            & m \land h \\
             \hline
            &\therefore c \lor g
        \end{aligned}
    \end{equation*}
    \item[] \textbf{Validate Argument: }
    \begin{proof}
    \begin{equation*}
        \begin{aligned}
            & 1. n \rightarrow \neg h & \text{Hypothesis} \\
            & 2. n \lor c & \text{Hypothesis} \\
            & 3. m \land h & \text{Hypothesis} \\
            & 4. n & \text{Assumption} \\
            & 5. \neg h & \text{Modus Ponens 1, 4} \\
            & 6. h & \text{Simplification 3} \\
            & 7. \bot & \text{Contradiction 5, 6} \\
            & 8. c & \text{Negation Introduction 4-7} \\
            & 9. c \lor g & \text{Addition 8} \\
            & 10. \boxed{c \lor g} & \text{Disjunctive Syllogism 2, 9} \\
        \end{aligned}
    \end{equation*}
    \end{proof}
\end{itemize}
\end{document}