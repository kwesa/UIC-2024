\documentclass{article}
\usepackage{fancyhdr}
\usepackage{amsmath}
\usepackage{amsthm}
\usepackage{accents}
\usepackage{amssymb}
\usepackage{graphicx}
\usepackage{float}
\usepackage{subcaption}
\usepackage{hyperref}
\usepackage{listings}
\usepackage{color}
\usepackage{tikz}
\usepackage{pgfplots}
\usepackage{pgfplotstable}
\usepackage{listings}
\usepackage{color}
\usepackage{tikz}
\usepackage{amsthm}
\usepackage{soul}
\usepackage[margin=15mm]{geometry}
\usepackage[makeroom]{cancel}

\definecolor{dkgreen}{rgb}{0,0.6,0}
\definecolor{gray}{rgb}{0.5,0.5,0.5}
\definecolor{mauve}{rgb}{0.58,0,0.82}
\hypersetup{colorlinks=true,linkcolor=blue, linktocpage}
% Set up fancy header
\pagestyle{fancy}
\fancyhf{} % Clear default header and footer
\rhead{Keith Wesa} % Right header
\lhead{CS 151 Homework 2:} % Left header
\rfoot{Page \thepage} % Right footer

\author{Keith Wesa}
\title{MAT 215 - Written Homework 1}
\date{\today}

\begin{document}
\section*{Question 2.3}
\subsection*{The problem}
\begin{itemize}
    \item[Q2.3] The difference of any rational number and any irrational number is irrational.
    \item[] \textbf{Define: } Rational Number: A number that can be expressed as the quotient or fraction $\frac{p}{q}$
    then $p$ and $q$ are integers and $q \neq 0$. 
    \item[] \textbf{Logical Statement:} $p \land q \rightarrow r$ where $r$ is the difference of $p$ and $q$. 
    \item[] \textbf{Argument Form: }
    \begin{equation*}
        \begin{array}{c}    
            p \land q \\
            \hline
            \therefore r
        \end{array}
    \end{equation*}
    \item[] \textbf{Thoughts on the problem: } We can prove this problem by contradiction. We can assume that the difference of any rational number and any irrational number is rational. 
    Then we can show that this assumption leads to a contradiction.
    \item[] \textbf{Validate Thoughts:}
    \begin{equation*}
        \begin{array}{c}    
            \lnot r \\
            \hline 
            \therefore \lnot p \lor \lnot q \text{ Modus Tollens}
        \end{array}
    \end{equation*}
\end{itemize}
\subsection*{Proof by Contradiction}
\begin{proof}
    \begin{itemize}
        \item[]Using the definition of rational numbers assume that the difference of any rational number and any irrational number is rational. Then there exists a rational number $p$ and an irrational number $q$ such that $p - q = r$ where $r$ is rational.
        \item[] \textbf{Defining Set: }$p, q, r \in \mathbb{R}$ and $p$ is rational and $q$ is irrational.
        \begin{align*}
            p - q &= r \\
            q &= p - r \\
            \text{Let } p &= \frac{a}{b} \text{ where } a, b \in \mathbb{Z} \text{ and } b \neq 0 \\
            \text{Let } r &= \frac{c}{d} \text{ where } c, d \in \mathbb{Z} \text{ and } d \neq 0 \\
            q &= \frac{a}{b} - \frac{c}{d} \\
            q &= \frac{ad - bc}{bd} \\
        \end{align*}
        \item[] Since $ad - bc$ and $bd$ are integers and $bd \neq 0$ then $q$ is rational. 
        \item[] This contradicts the assumption that $q$ is irrational. Therefore, the difference of any rational number and any irrational number is irrational.
    \end{itemize}
    \end{proof}
\end{document}