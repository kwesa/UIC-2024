\documentclass{article}
\usepackage{fancyhdr}
\usepackage{amsmath}
\usepackage{amsthm}
\usepackage{amssymb}
\usepackage{graphicx}
\usepackage{float}
\usepackage{subcaption}
\usepackage{hyperref}
\usepackage{listings}
\usepackage{color}
\usepackage{tikz}
\usepackage{pgfplots}
\usepackage{pgfplotstable}
\usepackage{listings}
\usepackage{color}
\usepackage{tikz}
\usepackage{amsthm}
\usepackage[margin=15mm]{geometry}

\definecolor{dkgreen}{rgb}{0,0.6,0}
\definecolor{gray}{rgb}{0.5,0.5,0.5}
\definecolor{mauve}{rgb}{0.58,0,0.82}

% Set up fancy header
\pagestyle{fancy}
\fancyhf{} % Clear default header and footer
\rhead{Keith Wesa} % Right header
\lhead{CS 151 Extra Credit 2:} % Left header
\rfoot{Page \thepage} % Right footer

\author{Keith Wesa}
\title{MAT 215 - Written Homework 1}
\date{\today}

\begin{document}
\section*{Extra Credit 2}
\subsection*{Problem}
\begin{itemize}
    \item[] let p be "I attended the lecture" q be "I watch the lecture recording", and r be "I do well on my quiz"
    \item[] Write the following argument in argument form (i.e using propositions and logical operators) and prove 
    that it is valid using the rules of inference. you must indicate the name of the rule used in each step of the proof
    \begin{align*}
        & \text{If I attend the lecture, then I will do well on my quiz} \\
            & \text{I attended the lecture or watch the lecture recording} \\
            & \text{I did not watch the lecture recording} \\
            & \text{Therefore, I will do well on my quiz} \\
    \end{align*}
    \item[]
    \item[] \textbf{Solution:}
    \begin{align*}
        & p \to r \text{ Hypothesis} \\
        &  p \lor q \text{ Not sure} \\
        & \lnot q \text{ Disjunctive syllogism}\\
        & \therefore r \text{ Disjunctive Syllogism}\\
    \end{align*}
\end{itemize}
\end{document}