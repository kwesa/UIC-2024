\documentclass{article}
\usepackage{fancyhdr}
\usepackage{amsmath}
\usepackage{amsthm}
\usepackage{amssymb}
\usepackage{graphicx}
\usepackage{float}
\usepackage{subcaption}
\usepackage{hyperref}
\usepackage{listings}
\usepackage{color}
\usepackage{tikz}
\usepackage{pgfplots}
\usepackage{pgfplotstable}
\usepackage{listings}
\usepackage{color}
\usepackage{tikz}
\usepackage{amsthm}
\usepackage[margin=15mm]{geometry}
\renewcommand\qedsymbol{$\blacksquare$}
\newtheorem*{theorem}{Theorem}
\newtheorem*{definition}{Definition}
\definecolor{dkgreen}{rgb}{0,0.6,0}
\definecolor{gray}{rgb}{0.5,0.5,0.5}
\definecolor{mauve}{rgb}{0.58,0,0.82}
\hypersetup{colorlinks=true,linkcolor=blue, linktocpage}
% Set up fancy header
\pagestyle{fancy}
\fancyhf{} % Clear default header and footer
\rhead{Keith Wesa} % Right header
\lhead{CS 151 Extra Credit 4} % Left header
\rfoot{Page \thepage} % Right footer

\author{Keith Wesa}
\title{MAT 215 - Written Homework 1}
\date{\today}

\begin{document}
\section*{Question 1}
\begin{itemize}
    \item[] Prove the following statement using a proof by contrapositive.
    \begin{itemize}
        \item[1.] For every non-zero real number $x$, if $x$ is irrational, then $\frac{1}{x}$ is irrational.
        \item[] 
        \item[] \textbf{Statement:} $p = \text{"x is irrational"}$, $q = \frac{1}{x} \text{"is irrational"}$
        \item[] \textbf{Contrapositive:} $\lnot p \rightarrow \lnot q$ $\equiv$ If $x$ is rational, then $\frac{1}{x}$ is rational
    \end{itemize}
    \begin{proof}
        \begin{theorem}
            For every non-zero real number $x$, if $x$ is irrational, then $\frac{1}{x}$ is irrational.
        \end{theorem}
        \begin{definition}
            A rational number is a number that can be expressed as the quotient or fraction $\frac{p}{q}$ of two integers, a numerator $p$ and a non-zero denominator $q$.
        \end{definition}
        \begin{definition}
            An irrational number is a number that cannot be expressed as the quotient or fraction $\frac{p}{q}$ of two integers, a numerator $p$ and a non-zero denominator $q$.
        \end{definition}
        \begin{itemize}
            \item[] \textbf{Let:} $x$ be a non-zero real number, $p$ and $q$ be integers, and $q \neq 0$.
            \item[] \textbf{Assume:} $x$ is rational.
        \end{itemize}
        \begin{align*}
            x &= \frac{p}{q} \\
            \frac{1}{x} &= \frac{q}{p} \\
        \end{align*}
        \begin{itemize}
            \item[] Since $p$ and $q$ are integers that can be expressed as a quotient of two integers $\frac{p}{q}$, then $\frac{1}{x}$ is rational given $x$ is a non-zero number.
            \item[] Since, $x$ is rational, then $\frac{1}{x}$ is rational.
            \item[] Therefore, for every non-zero real number $x$, if $x$ is irrational, then $\frac{1}{x}$ is irrational.
        \end{itemize}
    \end{proof}
\end{itemize}



\end{document}